\documentclass[letterpaper,12pt]{article}
\usepackage{array}
\usepackage{threeparttable}
\usepackage{geometry}
\geometry{letterpaper,tmargin=1in,bmargin=1in,lmargin=1.25in,rmargin=1.25in}
\usepackage{fancyhdr,lastpage}
\pagestyle{fancy}
\lhead{}
\chead{}
\rhead{}
\lfoot{}
\cfoot{}
\rfoot{\footnotesize\textsl{Page \thepage\ of \pageref{LastPage}}}
\renewcommand\headrulewidth{0pt}
\renewcommand\footrulewidth{0pt}
\usepackage[format=hang,font=normalsize,labelfont=bf]{caption}
\usepackage{listings}
\lstset{frame=single,
  language=Python,
  showstringspaces=false,
  columns=flexible,
  basicstyle={\small\ttfamily},
  numbers=none,
  breaklines=true,
  breakatwhitespace=true
  tabsize=3
}
\usepackage{amsmath}
\usepackage{amssymb}
\usepackage{amsthm}
\usepackage{harvard}
\usepackage{setspace}
\usepackage{float,color}
\usepackage[pdftex]{graphicx}
\usepackage{hyperref}
\hypersetup{colorlinks,linkcolor=red,urlcolor=blue}
\theoremstyle{definition}
\newtheorem{theorem}{Theorem}
\newtheorem{acknowledgement}[theorem]{Acknowledgement}
\newtheorem{algorithm}[theorem]{Algorithm}
\newtheorem{axiom}[theorem]{Axiom}
\newtheorem{case}[theorem]{Case}
\newtheorem{claim}[theorem]{Claim}
\newtheorem{conclusion}[theorem]{Conclusion}
\newtheorem{condition}[theorem]{Condition}
\newtheorem{conjecture}[theorem]{Conjecture}
\newtheorem{corollary}[theorem]{Corollary}
\newtheorem{criterion}[theorem]{Criterion}
\newtheorem{definition}[theorem]{Definition}
\newtheorem{derivation}{Derivation} % Number derivations on their own
\newtheorem{example}[theorem]{Example}
\newtheorem{exercise}[theorem]{Exercise}
\newtheorem{lemma}[theorem]{Lemma}
\newtheorem{notation}[theorem]{Notation}
\newtheorem{problem}[theorem]{Problem}
\newtheorem{proposition}{Proposition} % Number propositions on their own
\newtheorem{remark}[theorem]{Remark}
\newtheorem{solution}[theorem]{Solution}
\newtheorem{summary}[theorem]{Summary}
%\numberwithin{equation}{section}
\bibliographystyle{aer}
\newcommand\ve{\varepsilon}
\newcommand\boldline{\arrayrulewidth{1pt}\hline}
\newcolumntype{L}{>{\centering\arraybackslash}m{1.2cm}}
\doublespacing


\begin{document}

\begin{flushleft}
  \textbf{\large{Literature Review}} \\
  MACS 30200 \\
  Leoson Hoay
\end{flushleft}
\noindent{When Ludwig Wittgenstein first wrote in \textit{Philosophical Investigations}, which was posthumously published in 1953, that "Ethics and Aesthetics are one", the far-reaching implications of this statement (especially on other yet-to-be-invented fields of study) were unlikely to have been in his imagination. Many scholars since have dissected and debated this phrase regarding its claims on both ethics and aesthetics, but one thing is in general agreement - the link drawn by Wittgenstein between ethics and aesthetics is based on the fact that they both have to do with the idea of \textit{values} (Collinson 1985 and others). Indeed, humans perceive the environment with an eye for "beauty", and as much as one would like to ponder value as an instrinsic quality, the importance of aesthetics on at least the initial perception of value has been extensively elaborated upon and studied - qualitatively and experimentally - by scholars from art critic John Berger to psychologist Leslie Zebrowitz (Berger 1972, Zebrowitz 2008, Jacobsen 2010). As a philosopher who was mainly concerned with human communication, Wittgenstein's statement aptly resonates today within fields of study that seek to reconcile or delineate form, function, aesthetics and value. 

It follows that the communication of science, when taken as part of the grand collection of human communicative acts, also needs to concern itself with aesthetics. Studies in science communication that attempt to examine mechanisms and failures in communicating scientific knowledge have been very focused on embedded contexts, deficit models, and public science literacy (Nisbet and Scheufele, 2009). The deficit model posits that the interpretation of factual information occurs in the same way in audiences, and thus any failure of communication is a failure of the specific communicator in question, or public lack of literacy or irrationality (Bauer, 2008). While the deficit model is a widely upheld way of conceptualizing case studies in science communication, it is less helpful in certain aspects pertaining to the construction of metric-based solutions to communication failures. A chicken-and-egg problem exists with the alienation of the public as "not literate enough" or irrational (Nisbet and Scheufele, 2009), because this issue is exactly what better science communication is supposed to address. Moreover, this largely categorical framework does not accomodate for specific, metric-based analysis of the format of the communicative process. In this respect, an area that has not been given much attention is the aesthetics of the communicative artifact, and given the modern technological context, especially those artifacts that draw the attention of modern consumers of information; those platforms through which such individuals seek and receive information - the internet, and popular science media content creators.

Measuring aesthetics has been a largely interdisciplinary ground with limited contextual focus. 
}
\\
\\
\\
\\ 

\end{document}