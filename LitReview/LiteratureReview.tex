\documentclass[letterpaper,12pt]{article}
\usepackage{array}
\usepackage{threeparttable}
\usepackage{geometry}
\geometry{letterpaper,tmargin=1in,bmargin=1in,lmargin=1.25in,rmargin=1.25in}
\usepackage{fancyhdr,lastpage}
\pagestyle{fancy}
\lhead{}
\chead{}
\rhead{}
\lfoot{}
\cfoot{}
\rfoot{\footnotesize\textsl{Page \thepage\ of \pageref{LastPage}}}
\renewcommand\headrulewidth{0pt}
\renewcommand\footrulewidth{0pt}
\usepackage[format=hang,font=normalsize,labelfont=bf]{caption}
\usepackage{listings}
\lstset{frame=single,
  language=Python,
  showstringspaces=false,
  columns=flexible,
  basicstyle={\small\ttfamily},
  numbers=none,
  breaklines=true,
  breakatwhitespace=true
  tabsize=3
}
\usepackage{amsmath}
\usepackage{amssymb}
\usepackage{amsthm}
\usepackage{harvard}
\usepackage{setspace}
\usepackage{float,color}
\usepackage[pdftex]{graphicx}
\usepackage{hyperref}
\hypersetup{colorlinks,linkcolor=red,urlcolor=blue}
\theoremstyle{definition}
\newtheorem{theorem}{Theorem}
\newtheorem{acknowledgement}[theorem]{Acknowledgement}
\newtheorem{algorithm}[theorem]{Algorithm}
\newtheorem{axiom}[theorem]{Axiom}
\newtheorem{case}[theorem]{Case}
\newtheorem{claim}[theorem]{Claim}
\newtheorem{conclusion}[theorem]{Conclusion}
\newtheorem{condition}[theorem]{Condition}
\newtheorem{conjecture}[theorem]{Conjecture}
\newtheorem{corollary}[theorem]{Corollary}
\newtheorem{criterion}[theorem]{Criterion}
\newtheorem{definition}[theorem]{Definition}
\newtheorem{derivation}{Derivation} % Number derivations on their own
\newtheorem{example}[theorem]{Example}
\newtheorem{exercise}[theorem]{Exercise}
\newtheorem{lemma}[theorem]{Lemma}
\newtheorem{notation}[theorem]{Notation}
\newtheorem{problem}[theorem]{Problem}
\newtheorem{proposition}{Proposition} % Number propositions on their own
\newtheorem{remark}[theorem]{Remark}
\newtheorem{solution}[theorem]{Solution}
\newtheorem{summary}[theorem]{Summary}
%\numberwithin{equation}{section}
\bibliographystyle{aer}
\newcommand\ve{\varepsilon}
\newcommand\boldline{\arrayrulewidth{1pt}\hline}
\newcolumntype{L}{>{\centering\arraybackslash}m{1.2cm}}



\begin{document}

\begin{flushleft}
  \textbf{\large{Literature Review}} \\
  MACS 30200 \\
  Leoson Hoay
\end{flushleft}
\doublespacing
\fontfamily{ptm}\selectfont
\noindent{When Ludwig Wittgenstein first wrote in \textit{Philosophical Investigations}, which was posthumously published in 1953, that "Ethics and Aesthetics are one", the far-reaching implications of this statement (especially on other yet-to-be-invented fields of study) were unlikely to have been in his imagination. Many scholars since have dissected and debated this phrase regarding its claims on both ethics and aesthetics, but one thing is in general agreement - the link drawn by Wittgenstein between ethics and aesthetics is based on the fact that they both have to do with the idea of \textit{values} (Collinson 1985 and others). Indeed, humans perceive the environment with an eye for "beauty", and as much as one would like to ponder value as an instrinsic quality, the importance of aesthetics on at least the initial perception of value has been extensively elaborated upon and studied - qualitatively and experimentally - by scholars from art critic John Berger to psychologist Leslie Zebrowitz (Berger 1972, Zebrowitz 2008, Jacobsen 2010). As a philosopher who was mainly concerned with human communication, Wittgenstein's statement aptly resonates today within fields of study that seek to reconcile or delineate form, function, aesthetics, and value. 

Aesthetics, in terms of the assessment of "beauty" and "visual niceness" in stimuli, has been shown to have an effect on human attention, initial impressions, and behavior (Zebrowitz 2008, Wang and Pomplun 2012). It follows that the communication of science, when taken as part of the grand collection of human communicative acts, also needs to concern itself with aesthetics. Studies in science communication that attempt to examine mechanisms and failures in communicating scientific knowledge have been very focused on embedded contexts, deficit models, and public science literacy (Nisbet and Scheufele, 2009). The deficit model posits that the interpretation of factual information occurs in the same way across all individuals in an audience, and thus any failure of communication is a failure of the specific communicator in question, or public lack of literacy or irrationality (Bauer, 2008). While the deficit model is a widely upheld way of conceptualizing case studies in science communication, it is less helpful in certain aspects pertaining to the construction of metric-based solutions to communication failures. A chicken-and-egg problem exists with the alienation of the public as "not literate enough" or irrational (Nisbet and Scheufele, 2009), because this issue is exactly what better science communication is supposed to address. Moreover, this largely categorical framework does not accomodate for specific, metric-based analysis of the format of the communicative process. In this respect, an area that has not been given much attention is the aesthetics of the communicative artifact (Gombrich, 1960), and given the modern technological context, especially those artifacts that draw the attention of modern consumers of information; those platforms through which such individuals seek and receive information - the internet, and popular science media content creators. 

Measuring image and page aesthetics has been a largely interdisciplinary ground with limited contextual focus. In this study, the focus is on webpage aesthetics, specifically pages of science media articles. Non-computational measures such as the VisAWI (Moshagen and Thielsch, 2010, 2013) are widely used in User Experience (UX) and commercial contexts to evaluate user perceptions of website aesthetics, but these usually require a controlled lab setting and proper sampling of participants, resources that may not always be available for studies in science communication media. Moreover, the dimensions of these constructs are not necessarily directly translatable into aesthetic corrections. Fortunately, the development of these tools have also helped to inform and iterate alongside computational measures, which have developed largely within the fields of HCI and computer vision. The VisAWI, for example, assesses aesthetics on the four dimensions of Simplicity, Diversity, Colorfulness, and Craftsmanship. Another inventory by Jiang et. al (2016) categorized aesthetics into five factors: Unity, Complexity, Intensity, Novelty, and Interactivity. We can see these concepts intertwined in HCI literature and computational measures of aesthetics. In Ngo, Samsudin, and Abdullah (2000), 5 measures are outlined:}
\\
\\
\textbf{Screen Balance}: A measure of how well layout elements are centered on a page, based on weighted space.
\\
\textbf{Screen Equilibrium}: A measure of how well layout elements gravitate towards a center of origin.
\\
\textbf{Screen Symmetry}: A measure of how symmetrical the placement of layout elements are.
\\
\textbf{Screen Sequence}: A measure of how well the screen elements adhere to natural eye movements.
\\
\textbf{Screen Order and Complexity}: A derivative from the above measures and adapted from Birkhoff (1933).
\\
\\
These measures definitely have their counterparts in the previously mentioned non-computational dimensions and factors. In addition to Ngo's five, Rigau, Feixas, and Sbert (2007) add what is known as Kolmogorov Complexity and Shannon Entropy to the mix, both of which are pixel-level analyses, adapted from physical and algorithmic models, of order and complexity within an image. Machado et. al (2015) developed an artificial neural network which allowed measures of edge density, image compression, and to some extent colourfulness to predict human ratings of image complexity. A combination of these measures will be adopted in this study to analyse the pages of science web articles, although not all measures will be incorporated. For example, the relevance of image compression measlures may be diminished when we consider the overall composition of a webpage and how it is displayed and accessed. What measures are included in the final model will be further discussed in the Methods section.

There have been studies that have utilised predictive models that incorporate computational measures of aesthetics. One such study deployed an experimental framework where participants gave ratings on their first impressions on websites which varied on Colourfulness and Visual Complexity (Reinecke et. al, 2013). The collected data was then fed into a mixed-effects model that was used to predict ratings based on the above two measures of a page. While the model was successful to an extent, this study, like many others in the field of HCI and computer vision (Altaboli and Lin, 2011), was one that applied to a very general context of web page or a computer screen, but misses the nuances of things that applied to specific types of displays and pages - online science communication, for example. Moreover, Reinecke and her colleagues modeled the variables on ratings of first impressions, which is not the sole or paramount concern in information transmission. Yet, this study paves the way for the potential and legitimacy of building predictive models from computationally derived measures of aesthetics. 

In examining the aesthetics of a science web article or science journalism, one should be concerned with the aesthetics of the text itself in addition to the picture being displayed. After all, science media is hugely driven by the content, and any attempt at measuring its aesthetics should not neglect the content in favour of solely focusing on appearance. Here, text aesthetics does not refer to how the text is positioned in the layout or arranged in a space, but rather to stylistic concerns. Any textual form subject to human reading that is concerned to some extent with providing a level of engagement or enjoyment subjects itself to the critical eye of the beholder. This is obviously a key concern for popular science media and public science communication as a whole. Unfortunately, it remains difficult to computationally examine text stylistics and "well-written-ness". However, document-to-vector models and software packages such as Gensim (Rehurek 2011, Rehurek and Sojka 2010) provide researchers with one avenue of examining the "aesthetics" of a group of text artifacts - via their semantic distance from one another.  

What does semantic distance add to the concept of the aesthetics of text? In examining products from a single company, one large concern is the branding of the product (Malik et. al, 2013). In the case of media and text products, this can be conceptualized as stylistic consistency over time. 
\\ 
  
\end{document}